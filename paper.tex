\documentclass[11pt]{article}
\usepackage{acl2016}
\usepackage{times}
\usepackage{url}
\usepackage{latexsym}

%\aclfinalcopy % Uncomment this line for the final submission
%\aclfinalcopy

\def\aclpaperid{16} %  Enter the acl Paper ID here

%\setlength\titlebox{5cm}
% You can expand the titlebox if you need extra space
% to show all the authors. Please do not make the titlebox
% smaller than 5cm (the original size); we will check this
% in the camera-ready version and ask you to change it back.
\usepackage[noinline,nomargin,noindex,draft]{fixme}
\fxsetup{theme=color,mode=multiuser}
\FXRegisterAuthor{n}{note}{NOTE} 
%\nnote[inline]{this is how to make regular notes...}
%\nerror[margin]{} or \fxerror[inline]{reserved for final paper}

\newcommand\BibTeX{B{\sc ib}\TeX}
\usepackage{xspace}
\newcommand\wtv{\texttt{w2v}\xspace}

\newcommand{\mytitle}{bot.zen @ EmpiriST 2015 - A minimally-deep learning PoS-tagger \\
    (trained for German CMC and Web data)}
\newcommand{\mypdftitle}{\mytitle}
\newcommand{\mypdfsubject}{in Proceedings of the 10th Web as Corpus Workshop
    (WAC-X) and the EmpiriST Shared Task}
\newcommand{\mystitle}{bot.zen - A minimally-deep learning PoS-tagger}
\newcommand{\mykeywords}{PoS tagger, word empbeddings, word2vec, neural
    network, RNN, LSTM, EmpiriST 2015 shared task}
\newcommand{\mypdfkeywords}{\mykeywords}

%%%% hyper & options
\definecolor{darkred}{rgb}{0.5,0,0}
\definecolor{darkgreen}{rgb}{0,0.5,0}
\definecolor{darkblue}{rgb}{0,0,0.5}

\usepackage[final=true, pdfstartview=FitH]{hyperref}
\hypersetup{
    pdftitle={\mypdftitle},
    pdfauthor={egon w. stemle <egon.stemle@eurac.edu>},
    pdfsubject={\mypdfsubject},
    pdfkeywords={\mypdfkeywords},
    setpagesize=true,
    colorlinks=true,
    linkcolor=darkred,
    urlcolor=darkblue,
    citecolor=darkgreen
}

\title{\mytitle}

\author{Egon W.~Stemle \\
  EURAC research \\
  Bozen-Bolzano, Italy \\
  {\tt egon.stemle@eurac.edu}}

\date{}


%%%%%%%%%%%%%%%%%%%%%%%%%%%%%%%%%%%%%%%%%%%%%%%%%%%%%%%%%%%%%%%%%%%%%%%%%%%%%%
\begin{document} %%%%%%%%%%%%%%%%%%%%%%%%%%%%%%%%%%%%%%%%%%%%%%%%%%%%%%%%%%%%%
\maketitle

\begin{abstract} %%%%%%%%%%%%%%%%%%%%%%%%%%%%%%%%%%%%%%%%%%%%%%%%%%%%%%%%%%%%%
    This article describes the system that participated in the Part-of-speech
    tagging subtask of the \emph{EmpiriST 2015 shared task on automatic
    linguistic annotation of computer-mediated communication / social
    media}.

    The system combines a small assertion of trending techniques, which
    implement matured methods, from NLP and ML to achieve competitive results
    on PoS tagging of German CMC and Web corpus data; in particular, the system
    uses word embeddings and character-level representations of word beginnings
    and endings in a LSTM RNN architecture.  
    Labelled data (Tiger v2.2 and EmpiriST) and unlabelled data (German
    Wikipedia) were used for training.

    The system is available under the APLv2 open-source license.
\end{abstract}


\section{Introduction} %%%%%%%%%%%%%%%%%%%%%%%%%%%%%%%%%%%%%%%%%%%%%%%%%%%%%%%
\label{sec:intro}

Part-of-speech (PoS) tagging is an essential processing stage for virtually all
NLP applications.
Subsequent tasks, like parsing, named-entity recognition, event
detection, and machine translation, often utilise PoS tags, and benefit
(directly or indirectly) from accurate tag sequences.
However, frequent phenomena in computer-mediated communication (CMC) and Web
corpora such as emoticons, acronyms, interaction words, iteration of letters,
graphostylistics, shortenings, addressing terms, spelling variations, and
boilerplate%
~\cite{androutsopoulos2007,BernardiniBaroniEvert2008,beisswenger2013}
deteriorate the performance of PoS-taggers%
~\cite{giesbrecht2009,baldwin-EtAl:2013:IJCNLP}.

To this end, the EmpiriST shared task (ST) invited developers of NLP
applications to adapt their tokenisation and PoS tagging tools and resources
for the processing of written German CMC and Web data~\cite{empirist2016}. 
The ST was divided into two subtasks, tokenisation and PoS tagging, and
for each subtask two data sets were provided (see Subsection%
~\ref{subsec:empirist}).
The systems were evaluated by the organisers on raw data for the tokenisation
subtask, and on unlabelled but pre-tokenised data for the PoS tagging subtask
(both on the same approx.~14,000 tokens).

We participated in the PoS tagging subtask of the ST with our new
minimally-deep learning PoS-tagger:
We combine \texttt{word2vec} (\wtv) word embeddings (WEs) with a single-layer
Long Short Term Memory (LSTM) recurrent neural network (RNN) architecture;
strictly speaking, \wtv is \emph{shallow}. 
Therefore we call the combination with a single hidden layer
\emph{minimally-deep}.  
The sequence of unlabelled \wtv representations of words is accompanied by
the sequence of n-grams of the word beginnings and endings, and is fed into the
RNN which in turn predicts PoS labels.

The paper is organised as follows: We present our system design in
Section~\ref{sec:design}, the implementation in
Section~\ref{sec:implementation}, and its evaluation in
Section~\ref{sec:casestudies}. 
Section~\ref{sec:conclusion} concludes with an outlook on possible
implementation improvements.


\section{Design} %%%%%%%%%%%%%%%%%%%%%%%%%%%%%%%%%%%%%%%%%%%%%%%%%%%%%%%%%%%%%
\label{sec:design}

Overall, our design takes inspiration from as far back as~\newcite{Benello1989}
who used four preceding words and one following word in a feed-forward neural
network with backpropagation for PoS tagging, builds upon the strong foundation
laid down by~\newcite{collobert:2011b} for a NN architecture and learning
algorithm that can be applied to various natural language processing tasks, and
ultimately is a variation of
% achieved an accuracy of $0.95$ on the Brown Corpus.
~\newcite{santos2014learning} who trained a NN for PoS tagging, with 
character-level and WE representations of words. 
%On the Wall Street Journal (WSJ) portion of the Penn Treebank they achieve
%accuracy scores of $0.9732$. 


\subsection{Word Embeddings} %%%%%%%%%%%%%%%%%%%%%%%%%%%%%%%%%%%%%%%%%%%%%%%%%

Recently, state-of-the-art results on various linguistic tasks were
accomplished by architectures using neural-network based WEs.
\newcite{baroni-dinu-kruszewski:2014:P14-1} conducted a set of
experiments comparing the popular
\wtv~\cite{DBLP:journals/corr/abs-1301-3781,arXiv:1310.4546}
implementation for creating WEs to other distributional methods with
state-of-the-art results across various (semantic) tasks. 
These results suggest that the word embeddings substantially
outperform the other architectures on semantic similarity and analogy
detection tasks.
Subsequently,~\newcite{TACL570} conducted a comprehensive set of
experiments and comparisons that suggest that much of the improved results are
due to the system design and parameter optimizations, rather than the selected
method.  
They conclude that "there does not seem to be a consistent significant
advantage to one approach over the other".

Word embeddings provide high-quality low dimensional vector representations of
words from large corpora of unlabelled data, and the representations, typically
computed using NNs, encode many linguistic regularities and
patterns~\cite{arXiv:1310.4546}.


\subsection{Character-Level Sub-Word Information} %%%%%%%%%%%%%%%%%%%%%%%%%%%%

The morphology of a word is opaque to WEs, and the relatedness of
the meaning of a lemma's different word forms, i.e.~its different string
representations, is \emph{not} systematically encoded. 
This means that in morphologically rich languages with long-tailed frequency
distributions, even some WE representations for word forms of common
lemmata may become very poor~\cite{DBLP:journals/corr/KimJSR15}.

We agree with~\newcite{santos2014learning}
and~\newcite{DBLP:journals/corr/KimJSR15} that sub-word information is very
important for PoS tagging, and therefore we augment the WE representations with
character-level representations of the word beginnings and endings; thereby, we
also stay language agnostic---at least, as much as possible---by avoiding the
need for, often language specific, morphological pre-processing.


\subsection{Recurrent Neural Network Layer} %%%%%%%%%%%%%%%%%%%%%%%%%%%%%%%%%%

Language Models are a central part of NLP.
They are used to place distributions over word sequences that encode systematic
structural properties of the sample of linguistic content they are built from,
and can then be used on novel content, e.g.~to rank it or predict some feature
on it. 
For a detailed overview on language modelling research see~\newcite{mikolov2012}.

A straight-forward approach to incorporate WEs into feature-based
language models is to use the embeddings' vector representations as features.
%For an overview see, e.g.~\newcite{Turian:2010:WRS:1858681.1858721}.
Having said that, WEs are also used in neural network architectures, where they
constitute (part of) the input to the network. 

Neural networks (NNs) consist of a large number of simple, highly interconnected
processing nodes in an architecture loosely inspired by the structure of the
cerebral cortex of the brain~\cite{oreilly2000}.
The nodes receive weighted inputs through these connections and \emph{fire}
according to their individual thresholds of their shared activation function.
A firing node passes on an activation to all successive connected nodes.
During learning the input is propagated through the network and the output is
compared to the desired output. 
Then, the weights of the connections (and the thresholds) are adjusted
step-wise so as to more closely resemble a configuration that would produce the
desired output.
After all input cases have been presented, the process typically starts over
again, and the output values will usually be closer to the correct values.

RNNs are NNs where the connections between the elements are directed cycles,
i.e.~the networks have loops, and this enables them to model sequential
dependencies of the input.
However, regular RNNs have fundamental difficulties learning long-term
dependencies, and special kinds of RNNs need to be used~\cite{Hochreiter1991}; 
a very popular kind is the so called long short-term memory (LSTM)
network proposed by~\newcite{Hochreiter:1997:LSM:1246443.1246450}.


\section{Implementation} %%%%%%%%%%%%%%%%%%%%%%%%%%%%%%%%%%%%%%%%%%%%%%%%%%%%%
\label{sec:implementation}

We maintain the implementation in a source code repository at
\url{https://github.com/bot-zen/}.  
The version tagged as {\tt 0.9} comprises the version that was used to generate
the results submitted to the ST.
The version tagged as {\tt 1.0} is identical at its core but comes with
explicit documentation on how to download and install external software, and
how to download and pre-process required corpora.

Our system feeds WEs and character-level sub-word information into a
single-layer RNN with a LSTM architecture.
%Overall, we can benefit with this design not only from available labelled data
%but also from available general and domain-specific unlabelled data


\subsection{Word Embeddings} %%%%%%%%%%%%%%%%%%%%%%%%%%%%%%%%%%%%%%%%%%%%%%%%%

%When computing WEs we keep in mind that \newcite{TACL570} observed that one
%specific configuration of~\wtv, namely the skip-gram model with negative
%sampling (SGNS) 
%"is a robust baseline.  While it might not be the best method for every task,
%it does not significantly underperform in any scenario. Moreover, SGNS is the
%fastest method to train, and cheapest (by far) in terms of disk space and
%memory consumption".
%Coincidentally,~\newcite{arXiv:1310.4546} also suggest to use SGNS
We incorporates \wtv's original C implementation for learning WEs%
\footnote{\url{https://code.google.com/archive/p/word2vec/}}
in an independent pre-processing step, i.e.~we pro-compute the WEs.
Then, we use gensim%
\footnote{\url{https://radimrehurek.com/gensim/}}%
, a Python tool for unsupervised semantic modelling from plain text, to load
the data, and to extract the vector representations of the embedded words as
input to our NN.


\subsection{Character-Level Sub-Word Information} %%%%%%%%%%%%%%%%%%%%%%%%%%%%

Our implementation uses a \emph{one-hot encoding} with a few additional
features for representing sub-word information.
The one-hot encoding transforms a categorical feature into a vector where
the categories are represented by equally many dimensions with binary values.
We convert a letter to lower-case and use the sets of ASCII characters, digits,
and punctuation marks as categories for the encoding.
Then, we add dimensions to represent more binary features like
\emph{'uppercase'} (was upper-case prior to conversion), \emph{'digit'} (is
digit), \emph{'punctuation'} (is punctuation mark), \emph{whitespace} (is white
space, except the new line character; note that this category is usually empty,
because we expect our tokens to \emph{not} include white space characters), and
\emph{unknown} (other characters, e.g.~diacritics).
This results in vectors with more than a single \emph{one-hot} dimension. 
%
% - alpha-characters are mapped to lower-case one-hot + 'is-uppercase'
% - digits are mapped to one-hot + 'is-digit'
% - punctuation marks are mapped to one-hot + 'is-punctuation'
% - whitespace (ecxept '\n') characters are mapped to one-hot +
% 'is-whitespace'
% - unknowns have their own one-hot
% * '\n' is treated as new-token-character

\subsection{Recurrent Neural Network Layer} %%%%%%%%%%%%%%%%%%%%%%%%%%%%%%%%%%

Our implementation uses Keras, a minimalist, highly modular NNs library,
written in Python and capable of running on top of either TensorFlow or
Theano~\cite{chollet2015}. 
In our case it runs on top of Theano, a Python library that allows to define,
optimize, and evaluate mathematical expressions involving multi-dimensional
arrays efficiently~\cite{Theano2016}.

The input to our network are sequences of the same length as the sentences we
process.
During training we group sentences of the same length into batches.
Each single word in the sequence is represented by its sub-word information and
two WEs that come from two sources (see Section~\ref{sec:casestudies}).
%Note that the WEs of the domain-specific corpus may have only been derived
%from a small corpus.
Unknown words, i.e.~words without a WE, are mapped to a randomly generated
vector representation once, and this representation is reused later.
In Total, each word is represented by $1,800$ features: two times $500$ (WEs),
and ten times $80$ for two 5-grams (word beginning and ending).
(If words are shorter than 5 characters their 5-grams are zero-padded.)

This sequential input is fed into a LSTM layer that, in turn, projects to a
fully connected output layer with softmax activation function.
We use categorical cross-entropy as loss function and backpropagation in
conjunction with the RMSprop optimization for learning.  
At the time of writing, this was the Keras default---or the explicitly
documented option to be used---for our type of architecture. 


\section{Case Study} %%%%%%%%%%%%%%%%%%%%%%%%%%%%%%%%%%%%%%%%%%%%%%%%%%%%%%%%%
\label{sec:casestudies}

We used our implementation to participate in the EmpiriST 2015 shared task.
First, we describe the corpora used for training, and then the specific system
configuration(s) for the ST.
%% and subsequently, to run experiments with different data and parameters for
%%training (see Subsection~\ref{ssec:case_postempirist}).
%Following, a short description of the data we used for training.


\subsection{Training Data for \wtv and PoS Tagging} %%%%%%%%%%%%%%%%%%%%%%%%%%
\subsubsection{Tiger v2.2 (PoS)} %%%%%%%%%%%%%%%%%%%%%%%%%%%%%%%%%%%%%%%%%%%%%
\emph{Tiger v2.2}%
\footnote{\url{http://www.ims.uni-stuttgart.de/forschung/ressourcen/korpora/tiger.html}}
is version 2.2 of the TIGERCorpus~\cite{Brants2004} containing German newspaper
texts. 
The corpus was semi-automatically PoS tagged, and is one of the standard
corpora used for German PoS tagging. 
It contains 888,238 tokens in 50,472 sentences.
For research and evaluation purposes, the TIGERCorpus can be downloaded for
free.


\subsubsection{German Wikipedia (\wtv)} %%%%%%%%%%%%%%%%%%%%%%%%%%%%%%%%%%%%%%
\emph{de.wiki'15}%
\footnote{\url{http://www1.ids-mannheim.de/kl/projekte/korpora/verfuegbarkeit.html\#Download}}
are user talk pages (messages from users to users, often questions and advice),
article talk pages (questions, concerns or comments related to improving a
Wikipedia article), and article pages of the German wikipedia from 2015, made
available by the \emph{Institut f\"{u}r Deutsche
Sprache}\footnote{\url{http://www.ids-mannheim.de}}.
%% user talk pages
%% 378,576,711 toks
%% 15,343,904 s
%%
%% article talk pages
%% 446,551,142 toks 
%% 17,275,385 s
%%
%% article pages
%% 1,126,462,560 toks
%% 46,740,611 s
%%
% tokens: 378,576,711+446,551,142+1,126,462,560 = 1,951,590,413
% sentences: 15,343,904+17,275,385+46,740,611 = 79,359,900
The corpus contains 2 billion tokens 
(talk:379m, article talk:447m, article:1,1bn) 
in 79 million sentences 
(talk:15m, article talk:17m, article:47m), is well-sized for \wtv, and
also (partly) resembles or target data.
It is available under the CC BY-SA 3.0\footnote{Creative Commons
Attribution-ShareAlike 3.0 Unported, i.e.~the data can be copied and
redistributed, and adapted for any purpose, even commercially. See
\url{http://creativecommons.org/licenses/by-sa/3.0/} for more details.}
license.


\subsubsection{EmpiriST 2015 Data (PoS and \wtv)} %%%%%%%%%%%%%%%%%%%%%%%%%%%%%
\label{subsec:empirist}
\emph{empirist}%
\footnote{\url{https://sites.google.com/site/empirist2015/home/shared-task-data}}
is the CMC and Web data made available by the organizers of the ST.
It contains data samples from different CMC genres and samples from text genres
on the Web.
The training corpus contains 10,053 tokens and was PoS tagged by two annotators
(unclear cases were decided by a third person). 
The trial corpus contains around 3,600 tokens (2,100 CMC%
\footnote{For evaluation during the development phase we used
    \emph{empirist}-trial.
    Unfortunately, we found out only later that the CMC part of the trial data
    is also part of the training data, i.e.~for the CMC data our evaluation
    data was identical with the training data.}%
, 1,500 Web) and was PoS tagged by one annotator (without systematic error
checks%
%, i.e.~the accuracy of the PoS annotations is likely lower than on the
%training data and on the ST final evaluation data
).
See~\newcite{empirist2016} for more details.

%An overview of the used training data can be found in Table~\ref{tab:corpora}.


%\begin{table}[h]
%\begin{center}
%\begin{tabular}{l|r|r}
%\hline
%\multicolumn{1}{c}{} & \multicolumn{1}{c}{Tokens} & \multicolumn{1}{c}{Sentences} \\ \hline
%empirist       & $10*10^3$     & $0.6*10^3$    \\ \hline
%Tiger v2.2     & $0.9*10^9$    & $0.05*10^6$   \\ \hline\hline
%de.wiki'15   	& $2.0*10^9$    & $79*10^6$     \\ \hline
%%C4Corpus      & $$            & $$            \\ \hline
%%DECOW14       & $11.7*10^9$   & $600*10^6$    \\ \hline
%%deTenTen	    & $19.9*10^9$   & $1300*10^6$   \\ \hline
%\end{tabular}
%\end{center}
%\caption{\label{tab:corpora}Training data used for the EmpiriST 2015 shared
%    task. The PoS tags of the upper two corpora were used for training.
%    % (upper part) and additional data for the subsequent experiments (lower
%    % part).
%} \end{table}

%\emph{C4Corpus}%
%\footnote{\url{https://github.com/dkpro/dkpro-c4corpus}} is a full documents
%German Web corpus that has been extracted from CommonCrawl, the largest
%publicly available general Web crawl to date. 
%The data is available under the CreativeCommons license
%family\footnote{\url{https://creativecommons.org/licenses/}}. 
%See Habernal~\shortcite{Habernal.et.al.2016.LREC} for details about the corpus
%construction pipeline, and other information about the corpus.
%
%\emph{DECOW14}%
%\footnote{\url{http://corporafromtheweb.org/decow14/}} is the sentence shuffled
%German Web corpus by COW created with the 2014 technology of the COW
%initiative~\cite{SchaeferBildhauer2012}.
%The data can be used for purposes of academic research.
%% $11.7*10^9$ tokens 
%% $0.6*10^9$ (624,767,747) sentences
%
%\emph{deTenTen} Web corpus.
%%DeTenTen
%% $19.9*10^9$ (19,918,263,493) tokens
%% $1.3*10^9$ (1,309,696,220) sentences
%\cite{1120431}


\subsection{EmpiriST 2015 shared task} %%%%%%%%%%%%%%%%%%%%%%%%%%%%%%%%%%%%%%%
\label{ssec:case_empirist}

For the ST we used one overall configuration for the system, but we used three
different corpus configurations for training.
Consequently, we participated in the ST with three runs:
we used PoS tags from \emph{empirist} (run 1), from \emph{Tiger v2.2} (run 2),
and from both (run 3).  
For \wtv we trained a 500-dimensional skip-gram model on \emph{empirist} that
ignored all words with less than 3 occurrences within a window size of 10;
it was trained with negative sampling (value 5) and erroneously%
\footnote{%
    According to \wtv's author, technically negative sampling and
    hierarchical softmax can be combined but one should avoid combining them~%
    (see~\url{https://groups.google.com/forum/\#!topic/word2vec-toolkit/WUWad9fL0jU}).
    
    We had forgotten to deactivate an option in a data processing script.
}
also with hierarchical softmax.
We also trained a 500-dimensional continuous bag-of-words model on
\emph{de.wiki'15} that ignored all words with less than 25 occurrences within a
window size of 10; it was trained with negative sampling (value 3) and
erroneously also with hierarchical softmax.

The rational behind training the two models differently was that according to
\wtv author's experience%
\footnote{\url{https://groups.google.com/d/msg/word2vec-toolkit/NLvYXU99cAM/E5ld8LcDxlAJ}}
a skip-gram model "works well with small amount[s] of the training data, [and]
represents well even rare words or phrases", and a cbow model is "several times
faster to train than the skip-gram, [and has] slightly better accuracy for the
frequent words".
The other \wtv parameters were left at their default settings%
\footnote{\texttt{-sample 1e-3 -iter 5 -alpha 0.025} for skip-gram and
\texttt{-alpha 0.05} for continuous bag-of-words}.
% empirist.bin -cbow 0 -size 500 -window 10 -hs 1 -negative 5 -min-count 3 
% bigdata.bin -cbow 1 -size 500 -window 10 -hs 1 -negative 3 -min-count 25

To optimize the system's output we ran a simple grid search for three
parameters: the hidden LSTM layer's size, the dropout value for the projections
from the LSTM to the output layer during training, and the number of epochs
during training. 
The found values were size:$1024$, dropout:$0.1$, epochs:$20$.
%lstm_output_dims = [128, 256, 512, 1024*]
%dropouts = [0.1*, 0.25, 0.5, 0.75]
%nb_epochs = [5, 10, 20*]

\begin{table}[h]
\begin{center}
\begin{tabular}{l|r|r}
\hline
\multicolumn{1}{c}{} & \multicolumn{1}{c}{CMC} & \multicolumn{1}{c}{Web} \\ \hline
(1) \emph{empirist}             & $81.03$       & $86.97$ \\ \hline
(2) \emph{Tiger v2.2}           & $73.56$       & $89.73$ \\ \hline
(3) \emph{empirist}+\emph{Tiger v2.2} & $85.42$ & $90.63$ \\ \hline\hline
Winning Team                    & $87.33$       & $93.55$ \\ \hline
\end{tabular}
\end{center}
\caption{\label{tab:results}Official results of our PoS tagger for the three
runs on the EmpiriST 2015 shared task data.
} 
\end{table}

%The official results are listed in Table~\ref{tab:results}.

% 1: empirist_training (tagset:ibk)
% 2: tiger-2.2 (tagset:1999)
% 3: empirist_training + tiger-2.2 (tagset:ibk)
%Our evaluation on the EmpiriST trial data resulted in these accuracy scores:
%1:~$0.88$, 
%2:~$0.78$, 
%3:~$0.93$;
%and the overall ST evaluation was:
%1:~$0.84$,
%2:~$0.82$,
%3:~$0.88$.

%\subsection{Post-EmpiriST Experiments}
%\label{ssec:case_postempirist}
%As a post-EmpiriST addition to the results we wanted to (1)~remedy some errors
%and oversights in our initial pre-processing and training, (2)~optimize the
%learning parameters of the word embedding models and the neural network, and
%(3)~train word embedding models on other data.
%\fxnote[margin]{add (1)}
%
%The optimal learning parameters for the word embedding models were selected
%according to a comprehensive set of experiments and comparisons by
%Levy~\shortcite{TACL570}, and observations and suggestions by
%Mikolov~\shortcite{arXiv:1310.4546}. 
%For the post-EmpiriST experiments we used the skip-gram with negative-sampling
%training method (SGNS), with a negative-sampling value of $15$ (and without
%hierarchical softmax, cf.~Subsection~\ref{ssec:case_empirist}), ignored all
%words with less than $100$ occurrences within a window size of $10$, and a
%sub-sampling value of $10^{-5}$. 
%% cbow=0, negative=15, hs=0, window=10, sample=10^-5 
%% arXiv:1310.4546: -negative 15 -sample 0.00001
%% TACL570: alpha=0.75 and dyn=1-> fixed in word2vec algorithm;
%%There are quite a few differences between the skip-gram and the CBOW models.
%%However, if you have a lot of training data, their performance should be
%%comparable.%
%%\footnote{\url{https://groups.google.com/d/msg/word2vec-toolkit/NLvYXU99cAM/E5ld8LcDxlAJ}}
%
%As additional data to train word embedding models we obtained and processed
%\emph{C4Corpus}, \emph{DECOW14}, and \emph{deTenTen}.
%
%\fixme[inline]{The system is said to be trained on various widely available
%    corpora of German such as Tiger 2.2, DECOW148 etc. It is not discussed in
%    any detail and thus does not become clear why these corpora were selected
%    and especially what the systematic benefit of using these different data
%    sets was; more specifically, the paper does not systematically explore the
%    diversity of the data and their contribution to a learning improvement on
%    the system. A section with a systematic evaluation is missing and so is a
%    section with a more thorough discussion as to why this particular approach
%    was selected. It is suggested to the authors to provide more background on
%    these questions.}


\section{Conclusion \& Outlook} %%%%%%%%%%%%%%%%%%%%%%%%%%%%%%%%%%%%%%%%%%%%%%
\label{sec:conclusion}

We presented our submission to the EmpiriST 2015 shared task, where we
participated in the PoS tagging sub-task with fair results on the CMC data and
adequate results on the Web data. 
Still, our implementation, albeit following state-of-the art designs and methods, is
quite unpolished, and can certainly gain performance with more detailed tuning.
For example, adding special sequence start and sequence stop symbols to the
input is typically done as a pre-processing step, which might improve the
results at the beginning and the end of sentences;
or we might gain some performance by adding additional hidden layers to enable
the network to learn more intermediate abstractions.
A more profound design change could also help, e.g.~Recurrent Memory Network
are a novel recurrent architecture that have been shown to outperform LSTMs on
some language modelling tasks.
%and sentence completion tasks, while also allowing insights into the data from
%a linguistic perspective~\newcite{tran-bisazza-monz:2016:N16-1}.
Finally, for learning the word embeddings we could use different corpora, or
selectively extract parts from large web-corpora resembling---as much as
possible---the type of data that is to be tagged.


%\section*{Acknowledgments} %%%%%%%%%%%%%%%%%%%%%%%%%%%%%%%%%%%%%%%%%%%%%%%%%%
%The acknowledgments should go immediately before the references.  Do
%not number the acknowledgments section. Do not include this section
%when submitting your paper for review.


% References %%%%%%%%%%%%%%%%%%%%%%%%%%%%%%%%%%%%%%%%%%%%%%%%%%%%%%%%%%%%%%%%%
%\clearpage
%\nocite{*}
\small
\bibliographystyle{acl2016}
\bibliography{paper}
\normalsize

%\appendix %%%%%%%%%%%%%%%%%%%%%%%%%%%%%%%%%%%%%%%%%%%%%%%%%%%%%%%%%%%%%%%%%%%
%\section{Supplemental Material}
%\label{sec:supplemental}
%Supplement

\end{document}
